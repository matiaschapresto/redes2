\section{Experimentación}
\subsection{Starbucks}
Como primer experimento realizamos una captura en la red de fibertel zone disponible desde el Starbucks de Corrientes y Malabia aproximadamente a las 5 de la tarde un día de semana. En principio no estaba clara la cantidad de dispositivos que utilizaban la red, ya que si bien en el Starbucks se encontraban aproximadamente 20 personas,  la mayoría de ellos utilizando un dispositivo electrónico, no sabemos cuantas de ellas estaban conectadas a esta red ni si había más dispositivos fuera del local conectados a la red.\bigskip

Luego de escuchar la red durante 15 minutos (el script para escuchar paquetes permite configurar cuanto tiempo se desea medir), logramos capturar un total de COMPLETAR paquetes. En total, registramos paquetes de ¿4? protocolos distintos. Presentamos un gráfico mostrando para cada protocolo la probabilidad de que un paquete de la muestra sea de ese protocolo. Como hay grandes diferencias de magnitud entre la cantidad de paquetes del protocolo más frecuente con el resto, decidimos que era visualmente mas informativo mostrar los valores en escala logarítmica, es decir, mostrar $log(p(s_i))$ para cada protocolo $s_i$. Como las $p(s_i)$ están entre $0$ y $1$ este logarítmico siempre será negativo, por lo que nos parece más natural mostrar en su lugar $-log(p(s_i))$. Por último, como las entropía de la fuente ($H(S)$) es la esperanza de estos valores, decidimos agregarla como una columna al gráfico para poder comparar como se comporta la información que brinda cada protocolo con respecto de la media.